\documentclass[../summary.tex]{subfiles}

\begin{document}

\section{Buildings}

\subsection{Study guide}

Module 8 on Buildings sketches basic insights into the environmental impact of buildings and their role in a transition towards a more sustainable society.

\begin{itemize}
	\item You should clearly understand the concepts discussed in the module, be able to recognize examples, and be able to read the diagrams.
	\item Important concepts:
	      \begin{itemize}
		      \item role of buildings in reducing environmental impact
		      \item existing policy on energy performance of buildings
		      \item life cycle of a building
		      \item life cycle assessment
		      \item operational and embodied carbon
		      \item carbon footprint and ecological footprint
		      \item life cycle financial impact of buildings
		      \item possible solutions for building renovation
		      \item possible solutions to reduce the environmental impact of buildings
		      \item urbanisation
		      \item urban heat island
	      \end{itemize}
	\item It is not necessary to memorise the numerical values shown in the diagrams, but it is necessary to understand the meaning of the diagrams.
\end{itemize}

\subsection{Environmental impact of construction}
\subsubsection{Context}

\textbf{The building sector is responsible for 40\% of the energy use and 36\% of greenhouse gas emissions. The building sector hence plays an important role in climate change}. Besides the global impact of greenhouse gas emissions, they cause local effects as well. In cities, temperatures are rising and are typically higher than in suburban areas. This is called the \textbf{heat-island effect} and is caused by higher absorption of solar radiation by buildings and infrastructure. This can be avoided by integrating water and vegetation in the city. 55\% of the global population lives in cities and it is expected that this will increase to about 70\% by 2050, this is called \textbf{urbanisation}.
\\\\
To reduce the \textbf{impact of buildings on climate change}, important steps have been taken. In 2010, the \textbf{European Commission introduced a directive on the energy performance of buildings} that requires that all \textbf{new buildings have to be nearly-zero energy}. This resulted in a transition of buildings, where the buildings today are well insulated, air-tight, ventilated and are foreseen of at least a share of renewable energy to fulfil the nearly-zero-energy requirements.

\subsubsection{Main challenges}

A first important challenge in reducing environmental impact is \textbf{reducing the energy use of existing buildings}. Older buildings are not well insulated and require an in-depth renovation. To reach the European climate goals, the renovations should be \textbf{deep energetic renovations}. This means that replacing windows or adding roof insulation is insufficient. By re-using the demolished parts of the buildings, we can reduce the need for virgin resources. \textbf{Buildings should hence be seen as material banks}, enabling urban mining.
\\\\
A second challenge relates to the need to reduce \textbf{embodied impacts}, these are \textbf{greenhouse gas emissions due to the production, transport and end-of-life treatment of materials}. So far, the main efforts to reduce the impact of buildings focused on reducing the \textbf{operational energy use of buildings} (blue blocks on figure \ref{fig:embodied-operational-trends}). To further reduce the carbon footprint of buildings, it is important to \textbf{take into account the embodied impact} (red blocks on figure \ref{fig:embodied-operational-trends}). \\
If this impact is ignored, changes might be made that lower operational impact, but increase life cycle impact. For newly built nearly-zero energy buildings, the \textbf{embodied emissions have become as important as the operational emissions}. This is visualised by the dashed line on figure \ref{fig:embodied-operational-trends}. The majority of the embodied emissions occur at the moment of renovation or construction, these are called \textbf{upfront emissions}, so reducing those is more urgent.

\begin{figure}[H]
	\centering
	\includegraphics[width=0.6\linewidth]{../images/8-embodied-operational-trends}
	\caption{Global trends in embodied and operational life cycle emissions}
	\label{fig:embodied-operational-trends}
\end{figure}

\ \\
Another challenge relates to other \textbf{environmental pressures caused by the building sector, beyond climate change effects}. Resource extraction for construction materials, land use for buildings, fresh water use, fine dust emissions, plastic waste, are causing pressure on our planet in terms of e.g. biodiversity loss, respiratory effects, shortages in fresh water, and acidification of the ocean.
\\
For many of these resources, we are currently beyond the planetary boundaries. Or, in short, \textbf{our ecological footprint is beyond the Earth's carrying capacity}. There is hence an urgent need to further rethink the building sector to reduce our ecological footprint.

\subsubsection{Life Cycle Analysis}

There is no simple and straightforward solution available to solve the various challenges the construction sector is facing to reduce its environmental impact. A combination of steps and strategies will be required to enable a further transition of our buildings and cities.
\\
To avoid burden shifting in time and impacts, it is important to have insights into the consequences of measures from a life cycle perspective. This means that all \textbf{life cycle stages of buildings} are considered when assessing their environmental impact, from the \textbf{production and construction phase}, over the \textbf{use phase} till the \textbf{end-of-life phase}. For buildings, the use phase will remain a very important one, due to the relatively long service life of buildings.

\begin{figure}[H]
	\centering
	\includegraphics[width=0.68\linewidth]{../images/8-LCA}
	\caption{Life Cycle Analysis (LCA) of buildings}
	\label{fig:lca}
\end{figure}

\ \\
In \textbf{Life cycle assessment}, typically three areas of protection are assessed: the impact on depletion of resources, the impact on quality of ecosystems and the impact on human health. The results of an LCA are not limited to the carbon footprint of a building, but consists of the environmental footprint. This environmental footprint covers a whole range of relevant environmental effects, such as climate change, ecotoxicity, resource depletion, carcinogenic effects, etc.


\subsection{Residential stock}

Figure \ref{fig:life-cycle-environmental-impact} shows three types of residential buildings. For each, the life cycle environmental impact of four variants is shown. The first bar represents the impact of the original dwelling, without any renovation measures. The second is the same building, but built according to the energy performance standard of 2010 and the final two variants represent buildings that are even better insulated and for which other materials were chosen.
\\\\
The analysis shows that the \textbf{main driver of the environmental impact} of the \textbf{oldest buildings} in our stock is the \textbf{energy use for heating}.
\\
The life cycle impact of the \textbf{newly built variant} shows a much lower life cycle environmental impact due to a \textbf{reduction of the operational energy use}. The \textbf{upfront impact} has only \textbf{slightly increased}. This increase is hence largely compensated for by the reduced operational impact.

\begin{figure}[H]
	\centering
	\includegraphics[width=0.8\linewidth]{../images/8-life-cycle-environmental-impact}
	\caption{Life cycle environmental impact of different types of buildings}
	\label{fig:life-cycle-environmental-impact}
\end{figure}
\ \\
\textbf{Embodied (upfront) impacts} can be reduced by using less materials, choosing materials with a lower environmental impact (cleaner production), circular building (e.g. urban mining) and extending the service life of buildings.
\\\\
Comparing the various building types reveals that the building type and layout influence the life cycle impact of the building, even if the insulation level and airtightness are identical.
\\\\
Besides the environmental life cycle impact, \textbf{costs} are important to ensure affordability. The same analysis but from an economic perspective shows significant differences. Now, \textbf{operational energy use} is \textbf{not the main driver any more}.
The \textbf{main driver} for the costs is \textbf{the investment cost and the maintenance cost} due to high labour costs. This can be seen in figure \ref{fig:life-cycle-financial-impact}.


\begin{figure}[H]
	\centering
	\includegraphics[width=0.9\linewidth]{../images/8-life-cycle-financial-impact}
	\caption{Life cycle financial impact of different types of buildings}
	\label{fig:life-cycle-financial-impact}
\end{figure}
\ \\
When looking at broader scope, we include the impact of the land used by the building itself and used by the infrastructure required to access the building. In addition, the mobility of the residents is now included. The materials needed for roads, service systems and parking lots are also included.
\\\\
Comparing two different locations, city centre versus suburban area, shows that the mobility impact is (on average) much lower for people living in cities due to a lower use of car-transport. For a nearly-zero energy building, mobility is the main driver of the life cycle impacts. It can be concluded that \textbf{the location of buildings}, and hence urban planning, is \textbf{crucial when aiming for a reduction in the environmental impact} of the built environment.

\subsection{Sustainable construction}

Thanks to \textbf{EU policy, operational energy use of new buildings is well managed} and all stakeholders in the construction sector (architects and engineers, material producers, constructors) have enabled an important transition in the sector. The energy performance of newly built dwellings to date is much better than buildings that were built 15 years ago. The following \textbf{main challenges have been identified to further lower the environmental impact of new buildings}:
\begin{itemize}
	\item First, \textbf{reducing the impact of the materials} by A) well-thought design of buildings to reduce the material needed, B) producing and using materials with a lower environmental impact and C) circular building (high-value reuse and recycling of materials).
	\item Second, \textbf{reducing the impact of transport by sound urban planning}.
	\item Third, \textbf{reducing the land used by buildings} due to densification. This will lower the environmental footprint of buildings directly, and indirectly lower impacts due to increased public transport as public transport becomes affordable with higher densities.
\end{itemize}

\end{document}