% !TeX spellcheck = en_GB
\documentclass[../summary.tex]{subfiles}

\begin{document}

\section{Economics for sustainability}

\subsection{Study guide}

This chapter will be about the connection between economy and environment and the shortcomings and opportunities of economics for sustainable development. The study guide for this chapter of the MOOC was a complete mess and basically covered the whole transcript. That is why, even though this part of the summary is still based on the study guide, there won't be a structured overview of it.
\\\\
Typical questions for this chapter are:
\begin{itemize}
	\item Definitions, terminology: be able to explain the definition, give examples
	\item Graphs: be able to explain and interpret the graphs
	\item Principles: be able to explain, give examples
\end{itemize}

\subsection{Gross Domestic Product}
\label{sec:10-gdp}

The \textbf{Gross Domestic Product} or \textbf{GDP} is a measure of total production that is used to reflect the economic health of a country or region and has strongly increased since the end of the $\mathrm{19^th}$ century. Figure \ref{fig:GDP} shows this exponential growth.

\begin{figure}[htbp]
	\centering
	\includegraphics[width=1\linewidth]{images/10-GDP-increase.png}
	\caption{Increase in GDP since $\mathrm{19^{th}}$ century}
	\label{fig:GDP}
\end{figure}

\newpage
This increase in GDP went alongside a lot of other positive phenomena, such as an increase in life expectancy, an increase in literacy, a decrease in hunger, and so on. However, at the same time, we also saw a rise in environmental pressures, such as greenhouse gas emissions, global warming, plastic pollution, declining biodiversity, and many more.
\\\\
\textbf{Economic growth} is the annual percentage change of real GDP. Real GDP is the GDP corrected for inflation and price changes.
\\\\
There are multiple ways to define GDP. The first approach is the so-called \textbf{added value approach}. We define GDP as the value of all final goods and services that are produced in a given country in a given period of time. A second way to define GDP is the \textbf{income approach}, GDP can be defined as the sum of all the incomes that are earned by the owners of factors of production in a given year, in a given country. The factors of production are labour, capital and land.
\\\\
We should look at GDP as \textbf{a measure of production and of economic activity}. It was not intended to be a measure of prosperity, well-being, or happiness. So \textbf{a lot of things are not picked up by the concept of GDP}. A first example of something that is missing in the GDP is the distribution of wealth or the inequality. A second thing which is missing from the GDP is unpaid work. A third exclusion is changes in the quality of the environment.
\\\\
Figure \ref{fig:emissions-and-GDP-per-capita} shows the emissions and GDP per capita for the different countries. Looking at the graph, there is a correlation between GDP per capita and emissions per capita. Richer countries, have higher emissions per capita. We can also see that the big countries, like China and India, are on an intermediate level of emissions.

\begin{figure}[htbp]
	\centering
	\includegraphics[width=0.9\linewidth]{images/10-emissions-and-GDP-per-capita.png}
	\caption{\COtwo\ emissions per capita versus GDP per capita, 2018}
	\label{fig:emissions-and-GDP-per-capita}
\end{figure}

If we study the evolution of the emissions versus the GDP per capita during the $\mathrm{20^th}$ and $\mathrm{21^st}$ century, like depicted on Figure \ref{fig:emissions-and-GDP-per-capita-long-term}, we notice two different trends. Countries like China and India started at a low level of both emissions and GDP per capita, but they gradually increased over time and are still increasing. On the other hand, countries like Belgium and the United States started increasing at an intermediate level of emissions per capita, but then reached a peak and started decreasing. This pattern is observed for many developed economies and the idea behind it is the so-called Kuznets curve.

\begin{figure}[htbp]
	\centering
	\includegraphics[width=0.9\linewidth]{images/10-emissions-and-GDP-per-capita-long-term.png}
	\caption{\COtwo\ emissions per capita versus GDP per capita, 1950 to 2018}
	\label{fig:emissions-and-GDP-per-capita-long-term}
\end{figure}

\subsection{The Kuznets curve}

Like we mentioned in section \ref{sec:10-gdp}, the \textbf{Kuznets curve} is used to describe the pattern we see in the evolution of emissions and GDP per capita for countries with developed economies. The Kuznets curve is typically (similar to) an inverted U-shape (figure \ref{fig:emissions-and-GDP-per-capita-long-term}), meaning that initially countries have very low emissions per capita and GDP per capita. But then they start to economically develop, and they start to industrialise. During that industrialisation phase, we see rapidly increasing emissions and GDP per capita. At a certain time, there typically comes a moment when economies start to shift away from this heavy industry towards other activities, like services. This means that the emission intensity of that economy is gradually decreasing and the growth of their emissions is also decreasing. This can also be influenced by governments stepping in and developing environmental policies to combat pollution. This evolution of first \textbf{increasing, peaking, and decreasing emissions}, we call the Kuznets curve.
\\\\
However, it is important to note that this curve is a correlation, and not (entirely) a causation. Hence, we can't just simply say that the low-income countries just need to focus on growth in order to solve the pollution problems.
\newpage
\subsection{Production versus consumption based emissions}

We can make a distinction between production and consumption based emissions. Up until this moment, we have been looking at \textbf{production perspective emissions}, a measure of emissions from production within a given territory. However, this is not the full story because we can also consume goods that are imported from other economies. If we want to get a better representation of our emissions, we should also consider emissions from imported goods. This is done using the \textbf{consumption perspective}.
\\\\
If we look at Belgium, we can see that the consumption based emissions are very high, while the production based emissions are intermediate. For production-based countries like China, we see the opposite. If we calculate the absolute level of emissions, we get an overview of the actual emissions we are causing. We can call this our \textbf{carbon footprint}.

\subsection{Drivers of environmental impact}

If we want to take a look at the drivers for \COtwo\ emissions, we first start with the \textbf{I-PAT relationship}. This relationship says that environmental impact (I) is the product of 3 drivers: population (P), affluence or GDP per capita (A) and technology (T). A special case of this I-PAT relationship is the \textbf{Kaya decomposition} of greenhouse gas emission evolution. This decomposition makes a more sophisticated distinction between two technological drivers. Hence, it states the \COtwo\ emissions can be written as a product of 4 factors.
\\\\
Like previously mentioned, the first driver is \textbf{population}. The more people there are, the more goods and services are produced and the more \COtwo\ will be produced.
\\\\
The second driver is GDP per capita. This is a measure of \textbf{affluence}. When people get richer, they typically demand more goods and services which will also result in a rise of emissions.
\\\\
In terms of \textbf{technology}, the Kaya decomposition makes a distinction between energy intensity and \COtwo\ intensity of energy production. The \textbf{energy intensity} measures how much energy we need to produce one unit (for example \$1) of GDP. Unlike the previous factors, we expect the energy intensity to go down over time because our economy becomes more of a service economy instead of an industrial economy and because of technological progress.
\\\\
The \textbf{\boldmath \COtwo\ intensity of energy production} measures how much \COtwo\ is released by producing one unit of energy. Since we are currently decarbonising our energy production by investing in solar panels and wind energy, we also expect this to go down over time.
\\\\
Figure \ref{fig:Kaya-decomposition-world} shows the global evolution of these different factors. We can see that the emissions and the GDP grow at the same rate, so there is no decoupling between them. For Belgium, however, the curves are separating and there is \textbf{absolute decoupling}: GDP per capita is growing, while emissions are going down in absolute amounts. A \textbf{relative decoupling} would mean that emissions are growing, but at a slower rate than the GDP per capita. When we compare the Kaya factors of Belgium to the world, our population is growing a lot slower than the global population. This results in slightly decreasing \COtwo\ emissions since 1990. We can see this in Figure \ref{fig:kaya-decomposition-belgium}.

\begin{figure}[htbp]
	\centering
	\includegraphics[width=1\linewidth]{images/10-kaya-decomposition-worldwide.png}
	\caption{Kaya decomposition (worldwide)}
	\label{fig:Kaya-decomposition-world}
\end{figure}

\begin{figure}[htbp]
	\centering
	\includegraphics[width=1\linewidth]{images/10-Kaya-decomposition-belgium.png}
	\caption{Kaya decomposition (Belgium)}
	\label{fig:kaya-decomposition-belgium}
\end{figure}
\newpage
\subsection{Frameworks of sustainable development}

Brundtland defines \textbf{sustainable development} as a development that meets the needs of the present generation without compromising the ability of future generations to meet their own needs. In order to make the concept of sustainable development more operational, we will make a distinction between so-called weak and strong sustainability. But before we do this, we will first introduce the \textbf{different types of capital in economics}.
\\\\
A first concept of capital is \textbf{physical capital}. This is the physical infrastructure: roads, bridges, ports, canals, but also data networks etc.
\\\\
The second type of capital in economics stems from labour, specifically the number of people and their skills. We call this \textbf{human capital}.
\\\\
Furthermore, there is also \textbf{natural capital}. This type of capital is generally linked with our renewable and non-renewable sources and level of pollution.
\\\\
Lastly, we also consider the institutional or \textbf{social capital}. This is about the institutions, police, legal system, judges and so on in a society.
\\\\
Now, we are able to define \textbf{weak sustainability}. A development path is called weakly sustainable if the total sum of natural, human, physical and social capital is not decreasing over time. this allows for substitution between different types of capital. For example, it's possible that your natural capital is decreasing, but over-compensated by an increase in human capital.
\\\\
In \textbf{strong sustainability} we require that each of the four capital stocks on their own are non-decreasing over time. So in that case it's not possible to compensate a decrease in one type by an increase in another type. All of them have to be non-decreasing over time and there can be no erosion.
\\\\
A first practical example of sustainable development is the \textbf{Adjusted Net Savings} by the World Bank. This concept tracks the evolution of different capital stocks over time on a country level. They make a distinction between the changes in physical, natural and human capital and then make the sum of these changes. If that total sum is positive, we can say that the adjusted net savings are positive and that this country is on a sustainable development track. Clearly, this is an example of a weak sustainability concept because we add the changes in the different capital stocks together. So a decrease in one can be compensated by an increase in another one.
\\\\
Another example is the \textbf{doughnut economy concept} of Kate Raworth. It shows the balance between basic human rights and the maximum limits of sustainability. This concept will be explained in section \ref{sec:11-donut-economy}.

\newpage
\subsection{Externalities}

In economics, we speak of an \textbf{externality} if a production or consumption activity of an economic agent is affecting the consumption or production possibilities of another economic agent without full compensation being paid through the market mechanism.
\\\\
An example: a paper factory negatively impacting a nearby fish farm by dumping toxic chemicals and wastewater in the river, thereby poisoning all the fish. That is an example of a \textbf{production activity negatively affecting production possibilities}. Now imagine that downstream, people are living along the river. Because of the water pollution, the water smells bad and that is negatively affecting the consumption possibilities of the home-owners: they can't enjoy a nice barbecue in their garden. That is an example of a \textbf{production activity negatively affecting consumption possibilities}. This example also allows us to introduce the \textbf{compensation condition}. Imagine that the home-owner bought their house at a discount because they knew that it would be affected by this negative externality. So they paid less than what they normally would have for such a nice house. That price discount is, to some extent, compensating for the externality. If the discount is high enough, it would cancel out the externality, and we can no longer speak of an externality. That is why this last condition of compensation is important in the definition of an externality.
\\\\
Next, we can take a look at how \textbf{externalities influence} the principle of a normal market of \textbf{supply and demand} and how they can lead to market failure. \textbf{Market failure} means that the outcome of the market interaction is leading to an allocation that is not optimal for society.
\\\\
On figure \ref{fig:influence-externalities}, for a specific product, we plot the quantity on the horizontal axis and the price on the vertical axis. In a normal market of supply and demand, as price for a product goes up, less people will buy it. \textbf{Demand} is shown as a downward-sloping curve. Meanwhile, we'll assume that if more of that product is produced, the price per unit goes up.\footnote{The marginal private cost curve depends on the kind of product. For products that require materials to produce, such as food, it will go up. For electricity, it will be roughly horizontal until the capacity of the grid is reached, then it wil grow rapidly, because new infrastructure is expensive. For software, it will be nearly zero, because new copies are just digital files.} Thus, the \textbf{marginal private cost} or \textbf{supply} is an upward-sloping curve. If externalities are present, the \textbf{marginal societal cost} will be higher than the marginal private cost. The marginal cost to society is the amount a product \textit{should} cost if you consider the impact to society. This impact to society is given as the \textbf{marginal externality cost}, for example carbon emissions, dumping waste in the water etc.
\\\\
The optimal outcome for society is when market equilibrium occurs at the intersection of demand and marginal societal cost (point $\mathrm{E_{1}}$ on figure \ref{fig:demand-supply-externality}).
But in reality, without any government intervention or regulation, market equilibrium will occur at the intersection of demand and supply (point $\mathrm{E_{0}}$ on figure \ref{fig:influence-externalities}). This results in a quantity being sold that is too high ($\mathrm{Q_{0} > Q^{*}}$) for a price that is too low ($\mathrm{P_{0} < P^{*}}$) e compared to what is optimal for society. We can say that the free market prices are not properly reflecting the marginal external (climate change) costs of production, in other words we end up with a market failure. Keep in mind that this is very simplified.\footnote{In reality, none of these lines are straight. The supply curve will usually go down before it goes up, think about how it's cheaper per unit to buy in bulk. The curve ends up increasing because the materials you use are also subject to supply and demand and to produce more, you need more production capability, more overhead, etc.}

\begin{figure}[htbp]
	\centering
	\begin{subfigure}{.5\textwidth}
		\centering
		\includegraphics[width=1\linewidth]{images/10-demand-supply.png}
		\caption{Without externalities}
		\label{fig:demand-supply}
	\end{subfigure}%
	\begin{subfigure}{.5\textwidth}
		\centering
		\includegraphics[width=1\linewidth]{images/10-demand-supply-externalities.png}
		\caption{With externalities}
		\label{fig:demand-supply-externality}
	\end{subfigure}
	\caption{Influence of externalities on supply/demand market}
	\label{fig:influence-externalities}
\end{figure}

\newpage
\subsection{Public goods}

A \textbf{public good} is a good that is characterised by two crucial characteristics. Public goods are \textbf{non-rival}, meaning that the consumption of that good by an individual does not reduce the amount available for others to consume. A second characteristic of a public good is that it is \textbf{non-excludable}. It is not possible to prevent people from consuming a public good.
\\\\
We can consider \textbf{lower emissions as a public good}. If countries reduce their emissions, that will lead to lower concentrations of greenhouse gases in the atmosphere, and in the future that will lead to less climate change. So that's clearly a benefit. This lower degree of climate change is a public good in the sense that it is non-rival. If one country like Belgium enjoys a better climate that does not prevent China from enjoying a better climate. It's also non-excludable. Imagine that many countries lower their emissions, but the US doesn't, then the US can still enjoy the benefits of a better climate.
\\\\
This will lead to \textbf{free riding behaviour}. If an actor, a country in this case, knows that it cannot be prevented from enjoying the benefits of a public good once it is provided, that makes it very tempting to refuse to contribute.
\\\\
We can illustrate this incentive to free riding with an \textbf{example of two countries} that are thinking about joining and contributing to an international agreement of emission reduction: the USA and China. They both have the option to contribute or to free ride. This gives us the \textbf{payoff matrix} we can see in figure \ref{fig:payoff-matrix}. We can now discuss the 4 different possibilities from the point of view of the US.
\begin{itemize}
	\item China contributes
	      \begin{enumerate}
		      \item USA free rides: For the USA, this is the most desirable outcome. The climate is improved, and they didn't need to do or spend anything.
		      \item USA contributes: Best outcome for the climate, but less desirable than option 1 for the USA, because they could have spent nothing and still gotten an improved climate.
	      \end{enumerate}
	\item China free rides
	      \begin{enumerate}
		      \setcounter{enumi}{2}
		      \item USA free rides: Worst outcome for the climate, but the USA didn't need to do or spend anything.
		      \item USA contributes: The worst situation for the USA, they spent money and effort to improve the climate and China gets the same benefits without doing anything.
	      \end{enumerate}
\end{itemize}
If you ignore the climate consequences, the best outcome for the USA is to free ride, no matter what China does. This is a typical example of a \textbf{prisoner dilemma}, as we will see in section \ref{sec:social-dilemmas}.

\begin{figure}[htbp]
	\centering
	\includegraphics[width=0.9\linewidth]{images/10-payoff-matrix.png}
	\caption{The payoff matrix between the USA and China}
	\label{fig:payoff-matrix}
\end{figure}

\ \\
Given this urge to free ride, there is a strong reason to call for government intervention to prevent market failure. This works fine at the \textbf{national level}. We have national governments that produce a lot of public goods (think about defence and education, legal system, public infrastructure and so on). These public goods are financed with compulsory taxation, paid by every citizen. However, at \textbf{international level}, we don't have those strong governments. So all cooperation on the provision of global public goods has to come from voluntary contributions and voluntary agreements like the Paris Protocol. That is why it will be very challenging to overcome this free riding incentive on an international level.

\subsection{Elinor Ostrom}

An important name in this field of research is Elinor Ostrom. Her research focused primarily on the management of local commons (public goods). Her main contribution is that she shows that communities often succeed in managing commons in a sustainable way without government intervention.
\\
Instead of the classic dichotomy between the market and the government, Ostrom showed that there is often a third option in which the users of a natural resource themselves set up sophisticated decision-making and enforcement systems to resolve their conflicts of interest. In this sense, her work complements, or even refutes the classical view called ``the tragedy of commons'', which states that non-excludability inevitably leads to overexploitation of commons in the absence of government action.
\\\\
The numerous examples of successful and sustainable management of commons that Ostrom cites are usually about relatively small local communities such as a fishing village or a small farming community managing a common pasture or forest. In such communities, mechanisms, formal or informal or otherwise, often exist to enforce restrictions on the use of the natural resource. Those who do not comply with catch limits may risk being expelled from the community. And those who do not belong to the community may be denied access to the resource, sometimes by force. There are many examples of global commons with numerous potential users, such as the atmosphere to emit greenhouse gases or tuna stocks in international waters. For global commons, high coordination and transaction costs often prevent the emergence of effective management mechanisms. In such cases, the tragedy of the commons still looms.

\subsection{Environmental policy}

We can conclude that externalities and public goods can cause markets to fail. That's why there is a need for public intervention to restore optimality. There are different ways governments can intervene in markets: \textbf{regulation, prohibition, subsidies, taxes} (and combinations).

\subsection{Cost efficiency}
In order to explain the concept of cost efficiency, we will first introduce the concept of a marginal emission reduction cost curve, or a so-called MAC curve. Typically, there are many, many different ways to reduce emissions and they typically differ in terms of the costs that they entail, but they also differ in the emission reduction potential that they can bring. A \textbf{marginal abatement cost curve (MAC)} is a concept or a device that helps you to trade off these different projects against each other. The marginal abatement cost is the cost of reducing a unit of pollution.
\\\\
We can compare two different ways to reduce emissions for the example case of two companies. The first way is to impose \textbf{identical reductions} on both companies. Since the cost of the different reduction processes is not the same, the total cost for both companies will not necessarily be the same. In this example we can see on figure \ref{fig:reduction-identical} that the Green company has a cost of 80, while the Blue company only has a cost of 42. A better way would be to base the reduction on \textbf{marginal cost}. This way, the Blue company will need to do more effort than the Green company, but the cost will be the same for both companies. This will also lead to the lowest total cost. This way is shown on figure \ref{fig:reduction-marginal-cost}. In the next section, we will see how this can be implemented by a government.

\begin{figure}[htbp]
	\centering
	\begin{subfigure}{.5\textwidth}
		\centering
		\includegraphics[width=1\linewidth]{images/10-identical-reduction-for-companies.png}
		\caption{Identical reductions}
		\label{fig:reduction-identical}
	\end{subfigure}%
	\begin{subfigure}{.5\textwidth}
		\centering
		\includegraphics[width=1\linewidth]{images/10-reduction-based-on-marginal-cost.png}
		\caption{According to marginal cost}
		\label{fig:reduction-marginal-cost}
	\end{subfigure}
	\caption{Total cost when imposing two different ways for reduction}
	\label{fig:reduction}
\end{figure}
\newpage
\subsection{Price-based instruments versus legal standards}

A big difference between price-based instruments and (legal) standards is that \textbf{price based instruments} are more \textbf{cost-efficient}. This means that the total cost to society to reduce emissions to a given level is as low as possible.
\\\\
Legal standards, or \textbf{static efficiency} are in most cases not cost efficient because it's very difficult for a regulator to know of each company's individual technology options. Usually the regulator will use a sector average to define the emission standards. And so that leaves a lot of options for cost savings aside. This is similar to the identical reductions of figure \ref{fig:reduction-identical}
\\\\
A better way to reduce emissions is price-based, or \textbf{dynamic efficiency}. If you tax companies based on the amount of emissions, they would implement reductions that are cheaper than the amount of tax they save. Whenever cheaper technologies become available, companies will want to implement them to save money. Every time the company pays taxes, they are incentivised to find new solutions to reduce emissions. This is similar to the reduction according to marginal cost of figure \ref{fig:reduction-marginal-cost}.

\subsection{Green tax reform}

When a government introduces a carbon tax, we notice that the tax causes users to implement reduction projects. The government could use some of the carbon tax revenue to give subsidies to households to invest in solar panels on their rooftop, to help companies build wind turbines, or to invest in energy efficiency of their installations.
\\\\
A second possible use of the revenues of a carbon tax is to invest in social policy. That's important because we know that introducing a carbon tax is typically regressive. With regressive we mean that it can make the income distribution even more unequal. The reason for that is that energy from fossil fuels make up a larger share of the budget in poor households compared to rich households. So if we introduce a carbon tax that will be relatively worse for poor households.
\\\\
A third possible use of the carbon tax revenue is for labour market policies. In a lot of European countries we are seeing very high rates of labour taxes. This is giving a disincentive for people to go out working and for employers to create new jobs. So it is holding back a lot of economic activity.
\\\\
Combining a carbon tax with lower labour taxes is an example of a \textbf{double dividend} (a win-win situation). The first benefit or dividend is lower environmental pressure, lower emissions, a better environment because of the taxation of emissions. The second dividend is that we will see more jobs because of lower labour taxes.
\\\\
The general idea for green tax reform is to tax unwanted behaviour/outcomes (such as emissions) and not wanted behaviour/outcomes (such as labour).
\newpage

\subsection{Taxes versus subsidies}
Taxes discourage the bad and subsidies encourage the good alternative. On paper, both seem equally effective, but they are not. There are two rebound effects we notice. The \textbf{first rebound effect} is that subsidies are partly effective, but miss the underlying consumption activity. Think about giving subsidies for electric cars. This encourages buying a greener car, but doesn't discourage driving that car when you could take the bus instead. The \textbf{second rebound effect} is that subsidies save money that can be used on other consumption activities. For example a family saves money on their electric vehicle, and they use the extra money to go on a holiday.
\\\\
This is not the only problem with subsidies. Subsidies should trigger good investments that would not be taken without subsidies. In reality, they often save money on investments that people were going to do either way. This is the problem of \textbf{additionality}. Another downside to subsidies is \textbf{the Matthew effect}. Subsidies should mostly help the poorer part of society, but they often have a higher impact on the richer part. Furthermore, \textbf{subsidies are expensive} for governments, and often lead to higher taxes, especially on labour.
\\\\
Before we conclude if taxes are better than subsidies or not, we will take a look at existing environmentally harmful subsidies in our tax scheme. A first example of bad subsidies or bad tax exemptions are salary cars. Employees are given cars for nearly free (and with a tax exemption) from their employer, which is nice, but leads to a lot of additional traffic, traffic jams, accidents and emissions. Another example are the subsidies for non-sustainable agricultural activities. Getting rid of these subsidies would be a step in the right direction.
\\\\
Another step is a carbon tax improvement. Right now, different kinds of energy consumption are taxed in different ways because they are submitted to different tax systems. Because of this, 30\% of energy use is not taxed, while the rest is taxed at various (high) rates.

\subsection{Link with other challenges}
Since economics and sustainability are such broad topics, we can find a lot of links with the other domains we studied earlier.
\\\\
A first link we can find is one with \textbf{climate and biodiversity}. The market uses too much fossil fuels because their negative externalities are ignored and it does not provide enough effort for biodiversity because its positive externalities are not taken into account. Furthermore, the principle of freeriding makes greenhouse gas reduction very difficult.
\\\\
We can also find a link with \textbf{demography}. A better economy/higher GDP often leads to higher \COtwo\ emission per capita. It is difficult to avoid this.
\\\\
The third link we can find is the link with \textbf{buildings} and \textbf{mobility}. The market brings too much car traffic because negative externalities are not taken into account. Smart road pricing might be a solution for this problem. Secondly, subsidies for salary cars are an example of bad subsidy.
\\\\
Another link we can find is the one with \textbf{raw materials} and \textbf{circular economy}. The market leads to lots of material extraction because negative externalities are not taken into account. A return system for empty PET bottles is an example of a way to use taxes/subsidies to make people recycle.
\\\\
The last link we will cover is the link with \textbf{energy}. solar panel, battery and electric car subsidies suffer from the Matthew effect. Also, the energy of heating households is not taxed in a good way. The taxation of all energy carriers should be aligned with full social cost.

\end{document}