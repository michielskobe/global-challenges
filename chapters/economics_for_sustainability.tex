\documentclass[../summary.tex]{subfiles}

\begin{document}
	
	\section{Economics for sustainability}
	
	\subsection{Study guide}
	
	The study guide for chapter 10 of the MOOC was a complete mess. That is why, even though this part of the summary is still based on the study guide, there won't be an overview of it in this summary.
	\\\\
	Typical questions are:
	\begin{itemize}
		\item Definitions, terminology: be able to explain the definition, give examples
		\item Graphs: be able to explain and interpret the graphs
		\item Principles: be able to explain, give examples
	\end{itemize}
	
	\subsection{Gross Domestic Product}
	
	The Gross Domestic Product or GDP is a measure of total production that is used to reflect the economic health of a country or region and has strongly increased since the end of the 19th century. Figure \ref{fig:GDP} shows this exponential growth.
		
	\begin{figure}[htbp]
		\centering
		\includegraphics[width=1\linewidth]{images/10-GDP-increase.png}
		\caption{Increase in GDP since 19th century}
		\label{fig:GDP}
	\end{figure}
	
	\newpage
	This increase in GDP went together with a lot of other positive phenomena, such as an increase in life expectancy, an increase in literacy, a decrease in hunger, and so on. However, at the same time, we also saw a rise in environmental pressures, such as greenhouse gas emissions, global warming, plastic pollution, declining biodiversity, and many more. 
	\\\\
	Economic growth is the annual percentage change of real GDP. Real GDP means it's GDP measured in monetary terms but it's corrected for inflation and price changes.
	\\\\
	There are multiple ways to define GDP. The first approach is the so-called added value approach. We define GDP as the value of all final goods and services that are produced in a given country in a given period of time. A second way to define GDP is to look at incomes. All that value added in the GDP is giving rise to incomes, incomes to the persons who own the factors of production. Factors of production are labour, capital and land. So in the second approach to GDP, the income approach, GDP can be defined as the sum of all the incomes that are earned by the owners of factors of production in a given year, in a given country. 
	\\\\
	We should look at GDP as a measure of production and of economic activity. It was not intended to be a measure of prosperity or well-being or happiness. So a lot of things are not picked up by the concept of GDP. A first example of something that is missing in the GDP is a good idea about the distribution or the inequality. A second thing which is missing from the GDP is unpaid work. A third element that is not included in the GDP are changes in the quality of the environment.
	\\\\	
	Figure \ref{fig:emissions-and-GDP-per-capita} shows the emissions and GDP per capita for the different countries. Looking at the graph, we can see there seems to be an increasing relationship between GDP per capita and emissions per capita. Richer countries, with a higher GDP per capita, have higher emissions per capita. We can also see that the big countries, like China and India, are on an intermediate level of emissions. If we study the evolution of the emissions versus the GDP per capita during the 20th and 21st century, like depicted on Figure \ref{fig:emissions-and-GDP-per-capita-long-term}, we notice two different trends. Countries like China and India started at a low level of both emissions and GDP per capita, but they gradually increased over time and are still increasing. On the other hand, we have countries like Belgium and the United States that started increasing at an intermediate level of emissions per capita, but then reached a peak and started decreasing. This pattern is observed for many developed economies and the idea behind it is the so-called Kuznets curve
	
	\begin{figure}[htbp]
		\centering
		\includegraphics[width=0.9\linewidth]{images/10-emissions-and-GDP-per-capita.png}
		\caption{$CO_{2}$ emissions per capita versus GDP per capita, 2018}
		\label{fig:emissions-and-GDP-per-capita}
		
	\end{figure}
	
	\begin{figure}[htbp]
		\centering
		\includegraphics[width=0.9\linewidth]{images/10-emissions-and-GDP-per-capita-long-term.png}
		\caption{$CO_{2}$ emissions per capita versus GDP per capita, 1950 to 2018}
		\label{fig:emissions-and-GDP-per-capita-long-term}
		
	\end{figure}

	\subsection{The Kuznets curve}
	
	 Like we mentioned in the previous section, the Kuznets curve is used to describe the pattern we see in the evolution of emissions and GDP per capita for countries with developed economies. The Kuznets curve is typically an inverted U-shape of form (Figure \ref{fig:emissions-and-GDP-per-capita-long-term}), meaning that initially countries have very low emissions per capita and GDP per capita. But then they start to economically develop and they start to industrialize. During that industrialization phase, we see rapidly increasing emissions in per capita terms, but also rapidly increasing GDP per capita. At a certain time, there typically comes a moment when economies start to shift away from this heavy industry towards other activities, like, for instance, services. This means that the emission intensity of that economy is gradually decreasing and the growth of their emissions is also decreasing. This can also be influenced by governments that are stepping in and are developing environmental policies to combat pollution. This evolution of first increasing, peaking and then decreasing emissions, that is what we call the Kuznets curve. 
	 \\\\
	 However, it is important to note that this curve is a correlation, and not a causal relationship. Hence, we can't just simply say that the low-income countries just need to focus on growth in order to solve the solution problems.
	
\end{document}