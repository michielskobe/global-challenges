% !TeX spellcheck = en_GB
\documentclass[../summary.tex]{subfiles}

\begin{document}

\section{Climate}
\subsection{Study guide}
Module 1 talks about climate change and the role humanity plays in it. Make sure you understand the following:
\begin{itemize}
	\item Basic climate concepts
	\item The causes of climate change
	\item International agreements such as the Paris Agreement
	\item The influence of greenhouse gases and radiative forcing
	\item Carbon budget and emission pathways
	\item What we can do about climate change
	\item The impact of climate change on sea levels
\end{itemize}
\subsection{Climate trends and causes}
\subsubsection{Climate change over geological cycles}
We know climate change is partly a natural phenomenon because of the research done with Antarctic ice. Figure \ref{fig:antarctic-ice-records} clearly shows a natural cycle in temperatures over the last 800,000 years. Other evidence pointing to this conclusion can be found in landscapes which have been altered by moving glacial ice.
\begin{figure}[h]
	\centering
	\includegraphics[width=0.7\linewidth]{../images/1-antarctic-ice-records.png}
	\caption{Antarctic ice records}
	\label{fig:antarctic-ice-records}
\end{figure}

\subsubsection{Causes of climate change}
There are a few factors which cause the change in climate on earth:
\begin{itemize}
	\item Variations in the earth's orbit around the sun (eccentricity)
	\item The axis of the Earth from pole to pole is tilted compared to this plane of movements, this is time dependent (obliquity)
	\item Precession of the Earth.
\end{itemize}
Of these especially obliquity is important. It can lead to very hot summers and very cold winters without a lot of snowfall. This leads to a shrinkage in the ice coverage of the Earth which leads to ever warmer temperatures. This feedback loop amplifies itself (albedo feedback). Another factor is the physical place of land mass. If it is close to the poles, ice can easily form and reflect heat back into space. Volcanic explosions are known to have an influence as well due to their tendency to throw light blocking particles in the atmosphere. Lastly the activity of the sun itself plays a role in the climate of the earth.
\\\\
Us humans have had an impact since the industrial revolution by throwing tiny particles in the air due to burning fossil fuels, cooling the earth. This effect is overshadowed by the fact that we emit a lot of greenhouse gasses whilst we burn the same fossil fuels. An illustration of our impact on \COtwo\ levels can be seen in figure \ref{fig:co2-history}.
\begin{figure}[h]
	\centering
	\includegraphics[width=0.7\linewidth]{../images/1-co2-history.png}
	\caption{Historic \COtwo\ records}
	\label{fig:co2-history}
\end{figure}

\newpage
\subsubsection{Human cause}
Global temperatures have increased by \SI{1.1}{\degreeCelsius} since industrialisation. There are five lines of evidence that our greenhouse gas emission is responsible for this:
\begin{enumerate}
	\item We see that our activities with fossil fuel and cement produces about \SI{9}{\giga\tonne} of carbon per year, while the natural rate is only \SI{0.9}{\giga\tonne\per year}. Luckily for us, the earth absorbs about \SI{5.5}{\giga\tonne\per year} back into itself, otherwise \COtwo\ levels would be double of what they are right now.
	\item We also look at the isotopic ratio between C-13 and C-12 carbon. The gradual lowering of this ratio suggests a lot of plant derived carbon sources (fossil fuels) have been dumped in the atmosphere.
	\item The physical mechanism by which greenhouse gasses prevent infra-red radiation from escaping our atmosphere has been proved already in 1850.
	\item Climate models where factors can be individually adjusted also show a large effect of our activity.
	\item Radiative forcing is what happens when the amount of energy that enters the Earth's atmosphere is different from what leaves it. This, again, is influenced by greenhouse gasses.
\end{enumerate}

\subsubsection{Radiative forcing}
As discussed in the previous section, radiative forcing has an impact on our climate, but how does it work? The only way in which the earth can exchange energy is through radiation interactions with space. This is depicted in figure \ref{fig:radiation-exchange}. In the end, about 31\% of the solar radiation is reflected. This yields a net solar radiation of about \SI{235}{\watt\per\square\metre}. In reality this net radiation is counteracted by the natural infrared radiation of earth, balancing the entire system.
\\\\
By emitting too many greenhouse gasses, we have brought this balance to an end, resulting in a net increase of energy. To investigate the forcing effect we have, we can look at a concept called radiative forcing. This is a very important concept in the whole climate research and also very important for policy implications because with this metric, we can estimate the effect of different forcing factors.
\\
\begin{figure}[h]
	\centering
	\includegraphics[width=0.7\linewidth]{../images/1-radiation-exchange.png}
	\caption{Earth's radiation exchange}
	\label{fig:radiation-exchange}
\end{figure}
\newpage
So when we sum everything up over the time period 1750 till now, then we can also look at the average radiative forcing over that time period. The radiative forcing by carbon dioxide is the largest contributor - so that's why we talk about \COtwo\ a lot when we are talking about climate change - namely \SI{2.16}{\watt\per\square\metre}. Other well-mixed gasses also play an important role, so they also cause a positive radiative forcing. Ozone as well.
% We moeten potentieel wel nog is zien naar deze sectie want ik had het echt zwaar gehad.

\subsection{Climate targets and pathways}
\subsubsection{International agreements}
A very important milestone in the climate negotiation is the Paris Agreement of 2015. The goal of this convention is to stabilise greenhouse gas concentrations at a level that would prevent dangerous anthropogenic interference with the climate system by keeping the maximum warming under \SI{2}{\degreeCelsius}, preferably 1.5. This target is based on scientific research by the entire climate community worldwide called the Intergovernmental Panel on Climate Change (IPCC).
\\\\
The IPCC bases their recommendations on hundreds of papers and years of research of how the emission of greenhouse-gasses impacts the Earth. Since there are so many aspects to climate change, they have formed five subdivisions:
\begin{description}
	\item[Geographic] Preserving coral reefs, Arctic and indigenous people, mountain glaciers and biodiversity hotspots.
	\item[Weather] Extremes like heat waves, heavy rains, droughts, wildfires, coastal flooding.
	\item[People] Concern for impacts that disproportionately affect particular groups.
	\item[Monetary damage] Global scale degradation, loss of ecosystems and biodiversity on a global scale could lead to global damages and making the world more difficult to live in
	\item[Climate system] Concern for the large, abrupt and sometimes irreversible changes in the system. These are called tipping points as well.
\end{description}

\newpage
\subsubsection{Carbon budget}
To reach our goal of keeping climate change to a minimum, we need to gauge how much we need to achieve for that. For this reason, the carbon budget was introduced: it is a measure of how much \COtwo\ we can still emit without going over the limit. Part of this budget has already been spent because we calculate the budget from the start of our industrialisation.
\\\\
Calculating this figure is not easy, but we can take some short-cuts: temperature is linearly proportional to \COtwo\ emissions. This is clearly shown in figure \ref{fig:co2-temp}. This of course isn't conclusive evidence about the relationship and thus the prediction of our carbon budget can be wrong. A probability of 66\% (the very likely range) has been chosen, meaning that the carbon budget is defined in a way that we have a 66\% chance to stay within the temperature targets that have been defined by the Paris Agreement.
\begin{figure}[h]
	\centering
	\includegraphics[width=0.7\linewidth]{../images/1-co2-temp.png}
	\caption{Relationship between \COtwo\ and temperature}
	\label{fig:co2-temp}
\end{figure}
% Nog een deel over andere gassen er uit gelaten omdat het er uit ziet als een sidenote

\subsubsection{Pathways}
Because the exact time of emission is almost irrelevant, the IPCC has found four different pathways to reducing emissions. There are essentially 2 types: Strong reductions at the start with lower effort later down the line, and low effort now with (almost unrealistic) reductions later. They also rely very much on negative emissions, meaning that carbon is stored either in biosystems or in engineering systems
\\\\
All pathways rely heavily on the reduction of fossil fuels. We also will have to do some land management: agriculture, forestry and other land use (AFOLU). Thirdly we can use bioenergy with carbon capture and storage (BECCS). The first two pathways mainly rely on the reduction of fossil fuels, the fourth pathway makes heavy use of BECCS and the third pathway combines both strategies.
\\\\
If we look at our efforts to reduce our \COtwo\ use by looking at the Emission Gap report, we see that our policy has had an impact on our emissions. However, to conform with the Paris Agreement by 2023 we still have a gap of about \SI{12}{\giga\tonne} to fill. % Bijna even veel dus als die hun dikke moeder

\subsection{Climate mitigation}
\subsubsection{What can I do?}
An important resource of information is from the so-called drawdown project. Drawdown is the point in the future when levels of greenhouse gases in the atmosphere stop climbing and start to steadily decline. The drawdown project is the world's leading resource for climate solutions. Drawdown lists solutions based on effectiveness. So they estimate for a certain time horizon, in this case from 2020 to 2050, how much of the emission is either reduced or sequestered.
\\\\
For example, by reducing food waste to an absolute minimum we can save about \SI{90}{\giga\tonne} of emissions which is equal to about 2 years of emissions. Switching to plant rich diets would also help enormously in reducing the impact of agriculture and food on the climate. Other important changes are along the lines of managing refrigerants better, increasing forest coverage, transitioning to clean energy... .

\subsubsection{Individual action}
There are four main ways to help fight climate change as an individual:
\begin{enumerate}
	\item By using your democratic right, you can vote for political parties who take climate seriously. You can also participate in climate marches and activism to keep political pressure high.
	\item Reducing your own carbon footprint: you can do this by changing habits such as diet, energy use, transportation choices... The most effective starting point here is informing yourself.
	\item Planet friendly investments by opting out of funds investing in fossil fuels. Look for an ethical bank and inform yourself.
	\item Support an upscaling of the transition. If you can spread these ways to fight climate change and motivate others to do the same, then we can do it together.
\end{enumerate}

\subsection{What can we expect for the future}
\subsubsection{Climate scenarios}
Running complete simulations of a possible future is expensive and difficult, that is why we introduced representative concentration pathways or RCPs. They are labelled according to their radiative forcing: RCP 8.5 is \SI{8.5}{\watt\per\square\metre}, RCP 5 is \SI{5}{\watt\per\square\metre}, etc.
\\\\
Using RCPs we can make simpler models which reflect global warming. For example RCP 1.9 will get us to reaching our goals set on the Paris Agreement. RCP 2.6 is still likely to keep global warming under \SI{2}{\degreeCelsius}, but it's not guaranteed. Anything higher than RCP 2.6 will not reach the agreed goals. RCP 8.5 represents no change in climate policy and will have a large impact on our way of living.
\\\\
Looking at the impact of this warming, we see that the poles of the earth will heat the most. There is also a disproportionate heating effect between land and water. This is because it takes a huge amount of energy to heat water compared to heating land. Overall, wet regions will get even wetter and dry regions will get even dryer.

\newpage
\subsubsection{Highlight: sea level}
\paragraph{Causes and predictions}\mbox{}
\\
As water heats up, it expands. This will -- if climate change is not properly addressed -- make sea levels rise significantly. Especially low-lying countries will be impacted by this change in a permanent and catastrophic way. Water, which is stored on land in the form of glaciers for example, is also a big factor driving rising sea levels.
\\\\
Up until now, the oceans have absorbed about 90\% of excess heat due to climate change. This is a good thing for us, as it protects us from a lot of the negative effects of climate change. However, due to thermal expansion, the heating of the ocean leads to rising sea levels. Glaciers melting due to rising temperatures have also had quite the impact. Nearly every major glacier has significantly reduced in size over the last century.
\\\\
The change in ice sheets at the poles can't be underestimated either. Every way in which we monitor these sheets has shown a large reduction in their size. All the resulting water of course ends up in the oceans, raising their level. If for example, the entirety of Antarctica would melt, we'd see a rise of \SI{58}{\metre}(!).
\\\\
Current predictions estimate a rise of about 25 to \SI{100}{\centi\metre} in the coming century.

\paragraph{Impacts and risks of sea level rise}\mbox{}
\\
As is shown in figure \ref{fig:sea-rise-risk}, the risks for coastal regions are very substantial when nothing is done about climate change. At current levels, a risk is already present for resource rich coastal cities. We can mitigate this partly by adapting to the changes, but not everything will be salvageable.
\\
\begin{figure}[h]
	\centering
	\includegraphics[width=0.9\linewidth]{../images/1-sea-rise-risk.png}
	\caption{Burning ember diagram for sea level rise risks}
	\label{fig:sea-rise-risk}
\end{figure}

The World Bank estimates that damages associated with these risks will cost billions per year by the middle of this century. Especially since we have a tendency to build large cities near water. About 800 million people are at risk of facing the impact of rising sea levels and storm surges.

\end{document}