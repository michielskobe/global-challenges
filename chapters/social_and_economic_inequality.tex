\documentclass[../summary.tex]{subfiles}

\begin{document}
	
	\section{Social and economic inequality}
	
	\subsection{Study guide}
	
	Module 11 sketches basic insights into inequality both within and between countries. Make sure to understand:
	\begin{itemize}
		\item Inequality as a multi-dimensional concept: you can’t express inequality in only a single number or indicator
		\item Sources of inequality (according to philosophical literature): fate, personal choice, social circumstances + which is considered problematic
		\item Inequality between countries versus inequality within countries
		\item Main criticism of Philip Alston against the World Bank’s 
		definition of the International Poverty Line
		\item Different drivers of inequality:
		\begin{itemize}
			\item What the theory says about the effects of neoliberal globalization on the evolution of inequality, and why reality proved partially different
			\item The increasing concentration of capital: understand the differences between wealth inequality and income inequality. Understand what r > g means in this respect, and how it can influence the way societies function (concentration of power, land robbing)
			\item environmental and climate disruption: poor people often live in the most polluted areas and are at the same time more often dependent on their natural environment (grazing, farming,…)
		\end{itemize}
		\item Reasons why we should be concerned about inequality (also beyond the mere ethics): link with human rights, link with economic development, link with environmental sustainability, societal
		well-being, political stability
		\item Policy solutions (and the link to the drivers)
		\item Kate Raworth’s model of the donut economy 
	\end{itemize}
	
	\subsection{Inequality as a multi-dimensional concept}
	
	
	
	\subsection{Sources of inequality}
	
	
	
	\subsection{Inequality between versus within countries}
	
	
	
	\subsection{Main criticism of Philip Alston}
	
	
	
	\subsection{Drivers of inequality}
	
	The three biggest drivers of inequality that played a key role in the past decades and still mater today are the neoliberal globalisation model, the increasing concentration of capital, and environmental and climate disruption.
	\\\\
	The \textbf{neoliberal globalization} has a strong economic underpinning, based on free trade as an engine of global prosperity. In open competitive world markets, whoever can produce a good or service cheaper than others will see their exports grow. Likewise, whenever imported goods are cheaper than those that are produced locally, consumers will benefit from lower prices under free trade and see their purchasing power increase. In a diverse and globalised world, all countries will tend to specialise in their most competitive sectors and reduce their production of other ones, which results in efficiency gains, economic growth and less poverty on a global scale. 
	
	\subsection{Concerns about inequality}
	
	
	
	\subsection{Policy solutions}
	
	
	
	\subsection{Kate Raworth's model of the donut economy}
\end{document}